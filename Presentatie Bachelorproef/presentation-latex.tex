% KU Leuven latex presentation template
%
% © 2012 Michael Hofmann
%
% This work is licensed under the Creative Commons Attribution 3.0 Unported License.
% To view a copy of this license, visit
% http://creativecommons.org/licenses/by/3.0/ or send a letter to Creative
% Commons, 444 Castro Street, Suite 900, Mountain View, California, 94041, USA.

\documentclass[t,12pt,english
\ifx\beamermode\undefined\else,\beamermode\fi
]{beamer}
%\setbeameroption{show notes}
%\setbeameroption{show only notes}

\input{configuratie.tex}

\title{Ontwikkeling van een\\autonoom lijn-volgend\\racevoertuig}
\author{\mbox{De Bruycker Jorik} \and \mbox{Bolle Jonas}}
\date{09/05/18}
\institute{3ELICTE}

\begin{document}

\setbeamertemplate{background canvas}[title]

\begin{frame}[plain,noframenumbering]
    \titlepage
\end{frame}

\usedefaultcanvas

\emptyfooter
\begin{frame}[noframenumbering]{Inhoud}
        \tableofcontents
    \end{frame}
\largefooter

\section{Ontwerp custom Arduino-board}\label{sec:pcb}

\begin{frame}{Eigen Eagle-ontwerp}
\begin{figure}[H]
	\centering
	\includegraphics[width=\textwidth]{eigenschematic.png}
\end{figure}
\end{frame}

\begin{frame}{Maken van de PCB}

\end{frame}

\section{Infrarood-sensoren}
\begin{frame}{Titel}

\end{frame}
\section{Hall-sensor en snelheidsmeting}
\begin{frame}{Titel}

\end{frame}
\section{RFID reader}
\begin{frame}{Titel}

\end{frame}
\section{Bluetooth communicatie}
\begin{frame}{Titel}

\end{frame}
\section{Software flowchart}
\begin{frame}{Titel}

\end{frame}

\end{document}
