\section{RFID-lezer en libraries}
Voor de communicatie tussen de Arduino en de PN532 NFC-module konden we kiezen voor drie verschillende communicatie-interfaces: HSU, I\textsuperscript{2}C en SPI.
Bij voorkeur gebruiken we voor deze communicatie SPI, het voordeel hiervan is dat dit over het algemeen veel sneller verloopt ten opzichte van I\textsuperscript{2}C-communicatie. Bij het gebruik van SPI traden er echter problemen op met het detecteren van de PN532 aangezien de gedownloade bibliotheken verouderd waren.
HSU gebruikt dan weer pinnen D0 en D1, die we tijdens het prototypen ook gebruikten om met behulp van de seri\"ele monitor te debuggen, bijgevolgd konden we deze communicatie-interface ook niet gebruiken. Uiteindelijk kozen we er dus toch voor om gebruik te maken van I\textsuperscript{2}C, wat wel als voordeel heeft dat we minder pinnen hoeven te gebruiken.
\section{Arduino PCB}
Bij het maken van de custom Arduino is er een kleine opmerking bovengekomen waar we in het vervolg kunnen op letten. We hebben een te klein massa vlak voorzien onder de ontkoppelcondensator van onze 5V-regulator. Hierdoor kan de condensator zijn warmte niet goed dissiperen en wordt hij zeer warm. Al bij al heeft dit geen echte problemen opgeleverd, maar hoe warmer de condensator, hoe korter de levensduur van de component. In het vervolg letten we dus op de plaatsing van zo'n componenten in het massa vlak.