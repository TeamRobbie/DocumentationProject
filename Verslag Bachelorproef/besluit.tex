Gedurende de voorbije maanden ontwikkelden we een autonoom wagentje dat zo snel mogelijk een raceparcours kan navigeren. We voorzagen ook een aantal andere functionaliteiten zoals het meten van de snelheid, het inlezen van RFID-tags en data communiceren over Bluetooth. We zijn er dus in geslaagd om onze vooropgestelde doelstellingen te verwezenlijken.

Tijdens dit project hebben we meer ervaring op gedaan met het realiseren van een relatief ruime opdracht en leerden we veel bij over zowel hardware als software. Zo hebben we meer inzicht verworven in de werking en toepassingen van een microcontroller. Tevens leerden we zelf vereiste sensoren en andere hardware uitzoeken, al dan niet aan de hand van datasheets, en deze gebruiken om custom hardware te ontwikkelen voor onze eigen specifieke doeleinden. We raakten meer vertrouwd met de praktische werking van deze hardware en leerden hoe we deze software-matig kunnen laten samenwerken. Eveneens staken we meer op over veelgebruikte communicatie-interfaces zoals I\textsuperscript{2}C, UART en SPI, en de voor- en nadelen eigen aan deze interfaces. Het was ook zeer interessant om zelf eens Bluetooth-communicatie op te zetten en deze te configureren voor onze eigen toepassing. Voor het bevestigen van de sensor arrays en de magneten in de wielas deden we ook ervaring op met 3D-printen. Kortom was dit project dus een aangename en effectieve manier om meer kennis te verwerven en praktische ervaring op te doen omtrent elektronica ontwerpen en projecten managen.

Na realisatie van al onze doelstellingen kunnen we concluderen dat een aantal punten toch anders en/of beter konden. Zo kon het inlezen van de IR-sensoren vlotter verlopen indien we de uitgangsspanningen van deze sensoren hardware-matig digitaliseerden. Ook bij de PID-regeling van het wagentje is er nog ruimte voor verbetering. Het had ook nog mogelijk geweest om andere leuke functionaliteiten toe te voegen, zoals bijvoorbeeld de line-following modus van ons voertuig te over-riden en het wagentje te besturen met een joystick of via de Raspberry Pi. Ook het in kaart proberen brengen van het parcours en de positie van RFID-tags zou een mogelijke uitdaging geweest zijn mits er meer tijd ter beschikking was.