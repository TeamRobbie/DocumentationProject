In dit project ontwikkelden we een autonoom wagentje dat een raceparcours kan navigeren. We voorzagen ook een aantal andere functionaliteiten zoals het meten van de snelheid, het inlezen van RFID-tags en communiceren over Bluetooth.
Tijdens het project hebben we meer ervaring op gedaan met het realiseren van een relatief ruime opdracht en leerden we veel bij over zowel hardware als software. Zo hebben we meer inzicht verworven in de werking van een microcontroller. Tevens leerden we zelf vereiste sensoren en andere hardware uitzoeken, al dan niet aan de hand van datasheets, en deze gebruiken om custom hardware te ontwikkelen voor onze eigen specifieke doeleinden. We raakten meer vertrouwd met de praktische werking van deze hardware en leerden hoe we deze software-matig kunnen laten samenwerken. Eveneens staken we meer op over veelgebruikte communicatie-interfaces zoals I\textsuperscript{2}C, UART en SPI, en de voor- en nadelen eigen aan deze interfaces.\\
Kortom was dit project dus een aangename en effectieve manier om meer kennis te verwerven en praktische ervaring op te doen omtrent elektronica.

Na realisatie van al onze doelstellingen kunnen we concluderen dat een aantal punten toch anders en/of beter konden. Zo kon het inlezen van de IR-sensoren vlotter verlopen indien we de uitgangsspanningen van deze sensoren hardware-matig digitaliseerden. Ook bij de PID-regeling van het wagentje is er nog ruimte voor verbetering.