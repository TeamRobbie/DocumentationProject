Op het einde van de projectweek in Nieuwpoort kreeg iedere groep een coach toegewezen. Dit was op basis van enkele positieve en negatieve eigenschappen die we van elk groepslid moesten neerpennen. Het is de eerste keer dat dit gedaan werd voor het project. Er is ons nu ook gevraagd om een evaluatie van onze coach op te stellen.

Onze coach voor het project is Bert Cox. We hebben over het algemeen een positieve ervaring gehad met hem: we konden steeds bij onze coach terecht als we vragen hadden of als we ergens vast zaten in ons project. Zo zaten we als concreet voorbeeld een tijdje in de problemen met onze PID-regeling van ons voertuig en heeft Bert ons op de goede weg gezet door ons documentatie over PID-regeling voor te leggen. Vervolgens heeft hij ook enkele voorstellen gedaan omtrent de plaatsing van onze sensoren om de PID-regeling goed te laten werken. Tijdens de projectweek zelf heeft hij ook met ons samengezeten om enkele mogelijkheden te bespreken. Zo heeft hij het voorstel gedaan om te gaan werken met een multiplexer om zo analoge pinnen op onze Arduino uit te sparen. Het was ook op aanraden van onze coach dat we in onze planning zo snel mogelijk aan het ontwerp van de custom Arduino begonnen zijn. 

Langs de andere kant hadden we graag een iets actievere opvolging van ons project gehad van onze coach. Sommige coaches kwamen iedere woensdag eens langs om te kijken hoe hun groep vorderde met hun project. Dit was bij ons minder van toepassing, desalniettemin konden we wel steeds langs gaan met problemen en vragen. 